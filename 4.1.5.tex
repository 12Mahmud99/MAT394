\documentclass{article}
\usepackage{amssymb}

\begin{document}
\section*{4.1.5} (page 124)

\subsection*{(a) If a and b are consecutive integers, show that 
$a^2+b^2+(ab)^2$ is a perfect square.}

\subsection*{proof)}
Let $b=a+1$ and substituting into the original expression we arrive at:
$$a^2+(a+1)^2+(a^2+a)^2=a^2+a^2+2a+1+a^4+2a^3+a^2$$
$$a^4+2a^3+2a^2+a^2+2a+1=a^4+a^2+1+2a^3+2a^2+2a$$

We recognize that $(x+y+z)^2=x^2+y^2+z^2+2xy+2yz+2xz$ thus,
$$a^4+a^2+1+2a^3+2a^2+2a=(a^2+a+1)^2$$ which is a perfect square.

\subsection*{(b)If $2a$ is the harmonic mean of $b$ and $c$ (\textit{i.e.} $2a=2/(1/b+1/c)$), show that 
the sum of the squares of the three numbers $a$,$b$, and $c$ is the square of a rational number.}

\subsection*{proof)}
We are to prove that $$a^2+b^2+c^2=\frac{p^2}{q^2},p,q\in \mathbb{Z}, q\neq 0$$.
We substitute $2a=\frac{1}{(\frac{1}{b}+\frac{1}{c})}$ into the expression, and we get:
$$a^2+b^2+c^2=(\frac{1}{(\frac{1}{b}+\frac{1}{c})})+b^2+c^2$$

\end{document}
